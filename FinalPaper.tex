\documentclass{article}
\usepackage[utf8]{inputenc}
\usepackage{amsfonts, amsmath, amssymb, amsthm}
\usepackage[margin = 1in]{geometry}
\usepackage{graphicx}
\usepackage{wrapfig}

%%%THEOREM ENVIRONMENTS%%%
\newtheorem{theorem}{Theorem}
\newtheorem{lemma}{Lemma}
\newtheorem{corollary}{Corollary}
\newtheorem{prop}{Proposition}

\theoremstyle{definition}
\newtheorem{definition}{Definition}
\newtheorem{problem}{Problem}
\def\acts{\curvearrowright}
\DeclareMathOperator{\Stab}{Stab}

%%%RESOURCE STATEMENT%%%
\newcommand{\collaborators}[1]{\noindent\textit{For this problem set I collaborated with: #1}}
\newcommand{\resources}[1]{\noindent\textit{For this problem set I used the following books/websites: #1}}
\newcommand{\norm}{\unlhd}
\newcommand{\im}{$Im$}
\newcommand{\Aut}{$Aut$}



%%%%%%%%%%%%%%%%%%%%%%
%%%%%%%%%%%%%%%%%%%%%%
%%%%%%%%%%%%%%%%%%%%%%
%%%%%%%%%%%%%%%%%%%%%%
%%%%%%%%%%%%%%%%%%%%%%


\title{Reflection Groups}
\author{Aryan Arora, John Moon-Black, Hugh Shanno, \& Henry Sottrel}
\date{\today}
\begin{document}
\maketitle

%%%%%%%%%%%%%%%%%%%%%%
%%%%%%%%%%%%%%%%%%%%%%
%%%%%%%%%%%%%%%%%%%%%%
%%%%%%%%%%%%%%%%%%%%%%
%%%%%%%%%%%%%%%%%%%%%%

%\hline
%\collaborators{[No One]}

%\resources{[titles/links to resources]}
%\hline

%%%%%%%%%%%%%%%%%%%%%%
%%%%%%%%%%%%%%%%%%%%%%
%%%%%%%%%%%%%%%%%%%%%%
%%%%%%%%%%%%%%%%%%%%%%
%%%%%%%%%%%%%%%%%%%%%%

\section*{Introduction}

\noindent In this paper we aim to synthesize material from Chapter 2 of \emph{Groups, Graphs, and Trees} by Dr. John Meier to explain key concepts relating to reflection groups. In particular, we want to explore how reflections can form the basis for a number of interesting groups and motivate the study of reflection groups. Most Abstract Algebra students are familiar with reflections only as elements of dihedral groups or cyclic subgroups. For example, they have likely encountered the groups $D_4 = \{e, r, r^2, r^3, s\}$ and $\langle s \rangle = \{e, s\}$ where $s$ is a reflection. Students are also likely familiar with simple properties about reflection like the property that all reflections have order $2$. Building off this foundational knowledge, we aim to show how powerful of a tool reflections are in algebra and the surprising results that emerge from their study. \\

\noindent We are already familiar with dihedral groups generated by $\langle r,s \rangle$, where $s$ is a reflection across the center of the shape. Then, because $r$ is a rotation, we have been able to see that $r^n s$ is a reflection about some rotation $r^n$, because the point is flipped and then rotated $r^n$ times, so the point was effectively flipped over different line. In order to generate the reflection $r^n s$, we can see that $rs (s rs)^{n-1} = r^n s$ because $s$ is its own self inverse, so $(s rs)^{n-1} = sr^{n-1}s$. Now, when $k$ is finite, we can see that in $D_k$, $r^k = e$, so $rs (s rs)^{n-1}$ does have an order. However, in the case of $D_\infty$, the infinite dihedral group, we can see that $r^n \neq e$ for all $n \in \mathbb{N}$, so every $r^n s$ is a unique reflection. We can also see that $s$ and $rs$ are their own self inverses, as is required to be a reflection. Because we can generate every reflection in $D_\infty$ with $rs$ and $s$, as well as every rotation through the string $(rs s)^n = r^n$, we can see that $\langle s, rs \rangle$ is a reflection group in $D_\infty$.
\section*{One-Dimensional Reflection Groups}
\noindent \begin{wrapfigure}{r}{0.5\textwidth}
  \begin{center}
    \includegraphics[width=0.5\textwidth]{Prop 2.1 Drawing (4).pdf}
  \end{center}
  \caption{Composition of Reflections $a$ and $b$}
\end{wrapfigure}To build up an understanding of how we can combine reflections to form groups, we can begin with reflections in a single dimension. The simplest reflection is a reflection over a line. Consider, in a two dimensional Euclidean plane, parallel vertical lines at $x = 0$ and $x = 1$. Let $a$ be a reflection over the line $x = 0$ and $b$ a reflection over the line $x = 1$. For simplicity, we can think of $a$ and $b$ as functions. As figure 1 shows, the element $a$ maps any arbitrary element, $(x,y)$ to $(-x,y)$ while the element $b$ maps any arbitrary element, $(x,y)$ to $(2-x,y)$. Then, much like functions, we can compose reflections. The element $ab$ reflects the element $(x,y)$ first over the line $x = 1$ and then over the line $x = 0$ so the function $ab$ maps an element, $(x,y)\in\mathbb{R}^2$ to another element $(x-2,y)$ in $\mathbb{R}^2$. Similarly, $ba$ first reflects a point over the line $x = 0$ and then over the line $x = 1$ so $ba$ maps the element $(x,y)$ in $\mathbb{R}^2$ to $(x+2,y)$ in $\mathbb{R}^2$.\\
\\
Figure 1 demonstrates how compositions of reflections translate the axis of reflection. Consider the point $(0.5, 1)$. Then $a[(0.5, 1)] = (-0.5, 1)$ while $b[(0.5,1)]=(1.5,1)$. Similarly $ab[(0.5,1)] = (-1.5, 1)$  while $ba[(0.5,1)] =  (2.5, 1)$. Notice the point is only reflected in the $x$-direction while the $y$-coordinate is preserved.\\
\\
Figure 1 reveals an interesting pattern: left composing a reflection by $ab$ shifts the axis of reflection to the left by one and left composing the reflection by $ba$ shifts the axis of reflection to the right by one. For example, the axis of reflection for $a$ is $x = 0$ while the axis of reflection for $aba$ is $x= -1$ (a shift to the left by one). Similarly, the axis of reflection for $b$ is $x = 1$ while the axis of reflection for $bab$ is $x = 2$ (a shift to the right by one).\\
\\
Let $D_{\infty}$ be a group containing all finite compositions of $a$ and $b$. Then $a, b, ab, ba, aba, bab, ...$ are all elements of $D_{\infty}$. We now arrive at our first result about reflection groups,


\begin{theorem}
    A reflection over any vertical line $x = n$ where $n$ is an integer can be constructed from an element of $D_{\infty}$.
\end{theorem}


\begin{proof}
        We want to show that for any arbitrary vertical line $x = n$ where $n$ is an integer, we can generate an element, $d$, of $D_{\infty}$ such that $d$ reflects over the line $x = n$. We will prove this using induction for two cases: $n > 0$ and $n < 0$. We will first show that a reflection over the line $x = n$ can be generated by $(ba)^{n-1}b$ for all $n > 0$ and then we will show that a reflection over the line $x = n$ can be generated by $(ab)^{|n|}a$ for all $n < 0$.
\begin{lemma}
A reflection over any line $x = n$ where $n$ is an integer can be generate by either $(ba)^{n-1}b$ for $n > 0$ or $(ab)^{|n|}a$  for $n < 0$. \footnote{whenever used in this context, $|n|$ refers to the absolute value of $n$.} \\
\end{lemma}
\noindent We will prove this using induction.\\
\\
\textbf{Case 1: $n> 0$.}\\

\noindent Base case: $n = 2$. If $n = 2$, $(ba)^{n-1}b = (ba)^{1}b$. Let $(x,y)$ be an arbitrary point. So $bab[(x,y)] = (ba)(b[(x,y)])$. Because $b[(x,y)] = (2-x,y)$, $bab[(x,y)] = ba[(2-x,y)]$. Because $ba[(x,y)] = (x+2,y)$, $bab[(x,y)] = (4 - x, y)$. Then $bab$ is a reflection over the line $x = 2$. \\
    \\
    Assume the inductive hypothesis: A reflection over the line $x = n$ where $n > 0$ can be generated by $(ba)^{n-1}b$.\\
\\
    Inductive step: Let $x = n+1$ be a line of reflection. Note that $(ba)^{n}b = (ba)(ba)^{n-1}b$. By the inductive hypothesis $(ba)^{n-1}b$ is a reflection over the line $x = n$, and we previously established that $ba$ translates the line of reflection to the right by one. So $(ba)^{n}b$ is a reflection over the line $n + 1$, as desired.\\
\\
\textbf{Case 2: $n< 0$.} \\

\noindent Base case: $n = -1$. If $n = -1$,  $(ab)^{|n|}a = (ab)^{1}a = aba$. Let $(x,y)$ be an arbitrary point. Then $aba[(x,y)] = (ab)(a[(x,y)]$. Because $a[(x,y)] = (-x,y)$, $aba[(x,y)] = ab[(-x,y)]$. Because $ab[(x,y)] = (x-2,y)$, $aba[(x,y)] = (-2 - x, y).$ Then $aba$ is a reflection over the line $x = -1$.\\
\\
Assume the inductive hypothesis: A reflection over the line $x = n$ can be generated by $(ab)^{|n|}b$ for all $n < 0$.\\
\\
    Inductive step: Let $x = n-1$ be a line of reflection (recall that $n$ is negative so we are inducting away from the origin). Note that $(ab)^{|n-1|}a = (ab)(ab)^{|n|}a$. By the inductive hypothesis $(ab)^{|n|}a$ is a reflection over the line $x = n$, and we previously established that $ab$ translates the line of reflection to the left by one. So $(ab)^{|n-1|}a$ is a reflection over the line $n - 1$, as desired.  \\
\\
We have shown that any vertical line of reflection can be generated by the product of $a$'s and $b$'s. By construction of $D_{\infty}$, any element generated by the product of $a$'s and $b$'s is in $D_{\infty}$. Thus every vertical line $x = n$ is the axis of reflection for some element of $D_{\infty}$, as desired.
\end{proof}

\noindent Note that because the cardinality of the integers is infinite and we can produce a reflection over any a vertical line at $x=n$ where $n$ is any integer using an element of $D_{\infty}$, $D_{\infty}$ must be an infinite group. Additionally, note that to produce an infinite group, our basis reflections do not need to be vertical—just one unit apart. Later in this paper we prove that to generate the infinite group our two basis reflections need to be one unit apart, but for now it suffices to consider another group: $B_{\infty}$ which is generated by the finite compositions of the elements $m$ and $n$ where $m$ is a reflection over the line $y = -x$ and $n$ is a reflection over the line $y = -x + 2\sqrt{2}$. Refer to figure 2 for visual representation of reflection lines and note similarities to figure 1. 
\begin{wrapfigure}{r}{0.5\textwidth}
  \begin{center}
    \includegraphics[width=0.5\textwidth]{slanted Drawing.pdf}
  \end{center}
  \caption{Composition of Reflections $n$ and $m$}
\end{wrapfigure}

\noindent Returning to the group $D_{\infty}$ and synthesizing the points we have made so far, we can more generally assert that the function $(ab)^n$ maps an arbitrary element $(x,y)$ to $(x-2n, y)$ and that the function $(ba)^n$ maps an arbitrary element $(x,y)$ to $(x-2n,y)$. This implies a particularly important property about $D_{\infinity}$. 

\begin{theorem}
    A reflection over an arbitrary line $x = n$ where $n$ is an integer is generated by a single unique element of $D_{\infinity}.$
\end{theorem}
\begin{proof}
    Without loss of generality assume $n < 0$. Additionally assume that reflection over the line $x = n$ is generated by an element, $d$, of $D_{\infty}$ such that $d = (ab)^{|n|}a$. Towards contradiction, assume there exists another element, $f$, in $D_{\infty}$ such that $f$ also generates a reflection over the line $x = n$. \\
\\
    \textbf{Case 1: the final element of $f$ is $a$.} Assume that $f$ translates the basis reflection $a$ to create the reflection over x = n. Recall that left composing with $(ba)^{k}$ shifts the line of reflection to the right and left composing $(ab)^{k}$ shifts the line of reflection to the left. Because $a$ is a reflection over the line $x = 0$ and we have assumed that $n$ is negative, $f$ must left compose $a$ with $(ab)^{k}$. Then $d = (ab)^{|n|}a$ and $f = (ab)^{k}a$. But recall that each time we left compose $ab$ we shift the line of reflection to the left by one, so if $f \neq n$, $b$ and $f$ cannot have the same line of reflection. Then $k = n$ so $f = d = (ab)^{|n|}a$. \\
    \\
    \textbf{Case 2: the final element of $f$ is $b$.} Assume that $f$ translates the basis reflection $b$ to create the reflection over x = n. Recall that $a$ is a reflection over $x = 0$ and $b$ is a reflection over $x = 1$, so if $d = (ab)^{|n|}a$ generates the reflection over $x = n$, $f$ must be of the form $f = (ab)^{|n| + 1}b$. This can be rewritten as $(ab)^{|n|}(ab)b$. Because reflections have order 2, $(ab)^{|n|}(ab)b = (ab)^{|n|}(a)(bb) = (ab)^{|n|}(a)(e)$. Then $f = (ab)^{|n|}(a) = d$.\\
\\
    \noindent Then, in all cases if two elements of $D_{\infty}$ produce the same line of reflection, they must be the same element. Thus a reflection over an arbitrary line $x = n$ is generated by a single unique element of $D_{\infty}$.
\end{proof}
\noindent Implicit to this result is the idea that all elements of $D_{\infty}$ must begin and end with the same element (either $a$ or $b$) and must alternate between $a$ or $b$ throughout because any two $a$'s or $b$'s next to each other will produce the identity.\\
\\
While $D_{\infty}$ is an interesting group formed from reflections, many of its subgroups also have interesting properties. For example, the subgroup generated by $\langle a,bab \rangle$ has the interesting properties of having index 2 in $D_{\infty}$, or being normal in $D_{\infty}$, and being isomorphic to $D_{\infty}$. This property is shared with $\mathbb{Z}$ and certain subgroups, such as $2\mathbb{Z}$. At first it may seem unusual that this group shares a property with $\mathbb{Z}$ and $2\mathbb{Z}$ but recall that $D_\infty$ is the group of reflections over the lines $x=n$, where $n \in \mathbb{Z}$.\\
 
\begin{theorem}
    The subgroup of $D_{\infty}$ generated by $\langle a,bab \rangle$ has index 2 in $D_{\infty}$ and is isomorphic to $D_{\infty}$.
\end{theorem}

\begin{proof}
    We will first show that this subgroup is isomorphic to $D_{\infty}$ by defining and isomorphism between them, and then showing that the subgroup has index 2.\\
    \\
    Let $H$ be the subgroup of $D_\infty$ generated by $\langle a,bab \rangle$. Redefine $\alpha = a$, $\beta = bab$, and define $\varphi: \langle \alpha,\beta\rangle \rightarrow D_{\infty}$ where $\varphi:\alpha \mapsto a,\ \phi:\beta \mapsto b$.\\
    \\
    We will first show that $\varphi$ is a homomorphism. In order for $\varphi$ to follow the law of associativity of groups, $\varphi(x_1 \cdot x_2 \cdot x_3 \cdot ... \cdot x_n) = \varphi(x_1) \cdot \varphi(x_2) \cdot \varphi(x_3) \cdot ... \cdot \varphi(x_n)$ where $x_i \in \{\alpha, \beta\}$. Recall that $a$ and $b$ are self inverses, similar to $\alpha$ and $\beta$ since $\beta^2=(babab)(babab)=1$. Thus, $\varphi$ preserves the structure of its pre-image so $\varphi$ is well defined.\\
    \\
    Next, we will show that $\varphi$ is bijective. Let $y$ be an element of $D_\infty$. Note that since $y$ is in $D_\infty$, $y$ must be made up of $a$'s and $b$'s so by simply translating all the $a$'s to $\alpha$ and $b$'s to $\beta$, we get $\varphi^{-1}(y)$. Thus, $\varphi$ is surjective.  \\
\\
   Assume $\varphi(x_1)=\varphi(x_2)$. This yields that $\varphi$ translates $x_1$ and $x_2$ to the same element in $D_\infty$. Note that when simplified down, since $\varphi$ is a direct translation where $\alpha\mapsto a$ and $\beta\mapsto b$, $\varphi(x_1)=\varphi(x_2)$ if and only if $x_1=x_2$. Thus, $\varphi$ is injective, and thus, bijective. Thus we have concluded that $\varphi$ is an isomorphism so $\langle a,bab\rangle$ is isomorphic to $D_\infty$.\\
\\
    \textbf{Case 1: $x$ contains an odd number of $b$'s.} Assuming that $x$ contains an odd number of $b$'s, $x=aby$ or $x=by$ where $y\in\langle a,bab\rangle$. Note that $aby=b(bab)y\in b\langle a,bab\rangle$ and $by\in \langle a,bab\rangle$ by definition. Thus, $x\in b\langle a,bab\rangle$. \\
    \\
    \textbf{Case 2: $x$ contains an even number of $b$'s.} Assuming that $x$ contains an even number of $b$'s yields that $x$ must be generated by $a$'s and $bab$'s since $a$ and $b$ must alternate and if there exists an even number of $b$'s, then there must exist $\geq 0$ $bab$'s and no stranded $b$'s. Thus, $x\in \langle a,bab\rangle$.\\
    \\
    Thus we have shown that $\langle a,bab\rangle \cup b\langle a,bab\rangle=D_\infty$. In order to show that they are not pairwise disjoint cosets, we must simply show that there exist some elements in $b\langle a,bab\rangle$ that are not in $\langle a,bab\rangle$. Consider $b$. $b\in b\langle a,bab\rangle$ but is not in $\langle a,bab\rangle$.\\
    \\
    Thus $D_\infty/\langle a,bab\rangle$ contains only $2$ cosets, and thus, $\langle a,bab\rangle$ is a subgroup of index $2$.
\end{proof}
\noindent In fact, these properties are not unique to the subgroup generated by $\langle a,bab\rangle$. By using a similar isomorphism to $\varphi$ from theorem 3, we can further show that certain similarly generated subgroups in $D_\infty$ are also isomorphic 
 to $D_\infty$. For example, the subgroup generated by $\langle a,babab \rangle$ which is also isomorphic to $D_\infty$ and has index 3 in $D_\infty$.
\begin{theorem}
    The subgroup of $D_\infty$ generated by $\langle a,babab\rangle$ has index 3 in $D_\infty$ and is isomorphic to $D_\infty$.
\end{theorem}
\begin{proof}
    Since $a$ and $babab$ are also self inverses, we can define the same isomorphism $\varphi$ from theorem 3, but allowing $\alpha=a$ and $\beta=babab$. We have already in theorem 3 that $\varphi:\langle a,b\rangle \rightarrow \langle \alpha,\beta\rangle$ is an isomorphism.\\
    \\
    When examining the index of the subset $\langle a,babab\rangle$, the cosets are surprisingly similar to those in $D_\infty/\langle a,bab\rangle$ with the addition of $ab\langle a,babab\rangle$. Let $n$ denote the number of $b$'s in some $x$ in $D_\infty$.\\
    \\
\textbf{Case 1: $n=3k$ for some $k\in\mathbb{N}\cup \{0\}$.} If $x$ contains $3k$ individual $b$'s, $x$ must exist in $\langle a,babab\rangle$ since $babab$ contains $3$ $b$'s so it is possible to split $x$ up into $a$'s and $babab$'s. Further note that if $k=0$, then either $x=a$ or $x=0$ and $\{0,a\}\subseteq \langle a,babab\rangle$. Thus, $x\in \langle a,babab\rangle$.\\
\\
\textbf{Case 2A: $x$ starts with an $a$ and $n=3k+1$ for some $k\in\mathbb{N}$.} Thus, $x$ must be of the form $aby_1$ or $abay_2$ for some $y_1,y_2\in \langle a,babab\rangle$. Note that since $a\in H$, $abay_2=aby_2'$ where $y_2'=ay_2\in \langle a,babab\rangle$. Thus $x\in ab\langle a,babab\rangle$.\\
\\
\textbf{Case 2B: $x$ starts with a $b$ and $n=3k+1$ for some $k\in\mathbb{N}$.} Thus, $x$ must be of the form $by_1$ or $bay_2$ for some $y_1,y_2\in \langle a,babab\rangle$. Note that, similar to in Case 2A, since $a\in \langle a,babab\rangle$, $bay_2=by_2'$ for some $y_2'=ay_2\in \langle a,babab\rangle$. Thus $x\in b\langle a,babab\rangle$.\\
\\
\textbf{Case 3A: $x$ starts with an $a$ and $n=3k+2$ for some $k\in\mathbb{N}$.} Thus, $x$ must be of the form $ababy_1$ or $ababay_2$ for some $y_1,y_2\in \langle a,babab\rangle$. Note that $abab=b\cdot babab$ where $babab\in \langle a,babab\rangle$. Thus, $ababy_1=by_1'$ where $y_1'=bababy_1\in \langle a,babab\rangle$. Further note that $ababa=b\cdot babab\cdot a$ where $(babab\cdot a)\in \langle a,babab\rangle$. Thus, $ababay_2=b y_2'$ where $y_2'=bababay_2\in \langle a,babab\rangle$. Thus $x\in b\langle a,babab\rangle$.\\
\\
\textbf{Case 3B: $x$ starts with a $b$ and $n=3k+2$ for some $k\in\mathbb{N}$.} Thus, $x$ must be of the form $baby_1$ or $babay_2$ for some $y_1,y_2\in \langle a,babab\rangle$. Note that $bab=ab\cdot babab$ where $babab\in H$. Thus, $baby_1=aby_1'$ where $y_1'=bababy_1\in \langle a,babab\rangle$. Further note that $baba=ab\cdot babab\cdot a$ where $(babab\cdot a)\in \langle a,babab\rangle$. Thus, $babay_2=aby_2'$ where $y_2'=bababay_2\in \langle a,babab\rangle$. Thus $x\in ab\langle a,babab\rangle$.\\
\\
Thus we have shown that all of $D_\infty$ exists in the union of $\langle a,babab\rangle\cup b\langle a,babab\rangle\cup ab\langle a,babab\rangle$. Next, we will show that these cosets are different. Note that $a\in \langle a,babab\rangle$ while $a\notin (b\langle a,babab\rangle\cup ab\langle a,babab\rangle)$. Further note that $b\in b\langle a,babab\rangle$ but $b\notin (\langle a,babab\rangle\cup ab\langle a,babab\rangle)$. Thus $\langle a,babab\rangle$, $b\langle a,babab\rangle$, and $ab\langle a,babab\rangle$ are different cosets, and thus, $D_\infty/\langle a,babab\rangle$ contains only $3$ unique cosets, and thus, $\langle a, babab\rangle$ is a subgroup of index $3$ of $D_\infty$.
\end{proof}
\begin{center}\fbox{\begin{minipage}{15cm}\textbf{Example:} Consider the sequence $x=babababababab$. Note that $$babababababab=b(a)(babab)(a)(babab)$$ and $(a)(babab)(a)(babab)\in \langle a, babab\rangle$. Thus, $x\in b\langle a,babab\rangle$.\end{minipage}}\end{center}

\noindent In fact, we can see that for all $n \geq 1$, subgroups of the form $\langle a,(ba)^{n}b \rangle$ are isomorphic to $D_\infty$ and are of index $n+1$ in $D_\infty$.

\begin{theorem}
    For all $n \geq 1$, there exists a subgroup of $D_\infty$ that is isomorphic to $D_\infty$ and is of index $n+1$ in $D_\infty$
\end{theorem}

\begin{proof}
Define the subgroup $H$ to be of the form $\langle a, (ba)^{n-1}b \rangle$.
Let $\alpha = a$ and $\beta = (ba)^{n-1}b$.
We will define $\varphi: H \rightarrow D_\infty$ so that $\varphi(\alpha) = a$ and $\varphi(\beta) = b$, and so that for the arbitrary element $x_1,x_2,...,x_n$, where $x_i \in \{\alpha,\beta\}$, $\varphi(x_1,x_2,...,x_n) = \varphi(x_1)\varphi(x_2)...\varphi(x_n)$. We want to show that $\varphi$ is well-defined.
Let $x_1x_2...x_n = x_1'x_2'...x_n'$. We can see that since $\alpha$ starts and ends with $a$ and $\beta$ starts and ends with $b$ that if $x_1'x_2' = x_1x_2$ then $x_1 = x_1'$ and $x_2 = x_2'$, because otherwise one of the two would start or end with a different letter. We can then extrapolate this argument for all $x_i,x_i'$. Then we can see that $\varphi(x_1x_2...x_n) = \varphi(x_1)\varphi(x_2)...\varphi(x_n) = \varphi(x_1')\varphi(x_2')...\varphi(x_n') = \varphi(x_1'x_2'...x_n')$, so we can see that $\varphi$ is well-defined as desired.
Since $\varphi(\alpha) = a$ and $\varphi(\beta) = b$, we can see that $a,b \in \im\varphi$, and so because $D_\infty$ is generated from $a$ and $b$, we can conclude that $D_\infty$ is generated from $\im\varphi$, so $\varphi$ is surjective.\\
\\
Let $y = y_1,y_2,...,y_n$, where $y_i \in \{a,b\}$, because $D_\infty$ is generated from $a$'s and $b$'s. Then we can see that because only $\varphi(\alpha) = a$ that if $y_i = a$ then $\varphi(x_i) = a$ and so $x_i = \alpha$, and similarly that if $y_i = b$ then $\varphi(x_i) = b$ and so $x_i = \beta$. Thus, we have that there is only one pre-image for $y$, so $\varphi$ is injective as desired.\\
\\
Thus, $\varphi$ is an isomorphism from $H$ to $D_\infty$, and so $H \cong D_\infty$.\\
\\
Next, we want to show that $H$ is of index $n$ in $D_\infty$. We propose that all cosets of $D_\infty/H$ are $$H, bH, babH, ... (ba)^{n-2}bH.$$ We want to show that all of these cosets are distinct. We can see that because $e \in H$ that $b \in bH$, $bab \in babH$, ... $(ba)^{n-2}b \in (ba)^{n-2}bH$ as desired. Then, we can see that $b(ba)^{n-1}b = (ab)^{n-1}$, that $bab(ba)^{n-1}b = (ab)^{n-2}$, and further down that $(ba)^{n-2}b (ba)^{n-1}b = ab$, so we can see that the smallest elements that start with $a$ and $b$ in each group are distinct, so each coset is distinct as desired.\\
\\
Next, we want to show that $D_\infty = H \cup bH \cup ... \cup (ba)^{n-2}bH$. We can see that $e,a, (ba)^{n-1}b \in H$, that $b,(ab)^{n-1},ba \in bH$, that $bab, (ab)^{n-2},baba \in babH$, and so on until $(ba)^{n-2}b,ab \in (ba)^{n-2}bH$, so we can see that every element that starts with $a$ and $b$ and ends with $a$ and $b$ from $e$ to $(ba)^{n-1}b$ is represented in the cosets, and so we can see that the entirety of $D_\infty$ is represented in their union as desired. 
\end{proof}

\noindent Thus, we can conclude that for all $n \geq 1$ that $D_\infty$ contains a subgroup of index $n$ that is isomorphic to $D_\infty$. However, because these subgroups are of index $n$, we know that these subgroups do not induce an automorphism from $D_\infty$ to itself. This begs the question of what is required of a map $\alpha$ in order to be an automorphism?

\section*{Automorphisms in $D_\infty$}

\begin{center}\fbox{\begin{minipage}{15cm}\textbf{Example:} Define 
$\alpha: D_\infty \rightarrow D_\infty$ to be an automorphism, such that $\alpha(a) = b$ and $\alpha(b) = bab$. Then we can see that $\alpha(a)\alpha(a) = bb = e = \alpha(e) = \alpha(aa)$ as desired, and that $\alpha(b)\alpha(b) = babbab = e = \alpha(e) = \alpha(babbab)$ as desired, so $\alpha$ is well-defined, and is thus a homomorphism. Next, we can see that $\alpha(aba) = bbabb = a$, and that $\alpha(a) = b$, so we can generate every single element of $D_\infty$ from the image of $\alpha$, and so $\alpha$ is surjective. Lastly, we can see that for all $y_1y_2...y_n \in D_\infty$ that because only $\alpha(b)$ contains any $a$'s, and each $a$ must be properly positioned in $y_1y_2...y_n$ that there is only one possible input that produces each possible output, and so $\alpha$ is injective. Thus, we have that $\alpha$ is an automorphism in $D_\infty$. \end{minipage}}\end{center}

\noindent It turns out that in order for $\alpha$ to be an automorphism, $\alpha$ must be such that it takes $a$ and $b$ to be reflections fixing $x = n$ and $x = n+1$, for some $n \in \mathbb{Z}$.
\begin{lemma}\label{auto-fixation}
    If $\alpha$ is an automorphism of $D_\infty$ then $\alpha$ takes $a$ and $b$ to be reflections fixing the lines $x=n$ and $x=n+1$ for some $n \in \mathbb{Z}$.
\end{lemma}
%Declare notation r_n to be the reflection over the line x=n in this problem
\begin{proof}
    First of all, we know that $a$ and $b$ must map to reflections because the order of $a$ and $b$ are $2$, so $|\alpha(a)| = |\alpha(b)| = 2$ in order for $\alpha$ to be a homomorphism, and translations have infinite order.\\
\\
    Next, we will set $\alpha(a) = a$ and $\alpha(b) = r_n$, where $r_n$ is the reflection over the line $x=n$. We will proceed by proof on contradiction, and so we will assume that $n > 1$ because $a$ is the reflection over $x = 0$. Then we know that $\alpha(b) = (ba)^{n-1}b$. Then we can see that $\im \alpha = \langle \alpha(a), \alpha(b) \rangle = \langle a, (ba)^{n-1}b \rangle$. We want to show that $ab \notin \im \alpha$. Suppose that $ab \in \langle a, (ba)^{n-1}b \rangle$. Then we know that 
    \begin{align*}
            ab &= W_1W_2...W_n &\\
            &= aW_2...W_n &&\text{ because the first letter must be } a.\\
        &= a(ba)^{n-1}bW_3...W_n &&\text{ because the second letter must be a }b.
    \end{align*}
    Then we have that $ab = (a(ba)^{n-1}b)^k$ or $ab = (a(ba)^{n-1}b)^ka$.\\
    \textbf{Case 1: $ab = (a(ba)^{n-1}b)^k$.} We can see that $ab = (a(ba)^{n-1}b)^k = ((ab)^n)^k \neq ab$ because $n > 1$ and $k \geq 1$.\\
    \\
    \textbf{Case 2: $ab = (a(ba)^{n-1}b)^{k}a = (ab)^{nk}a$.} We can see that $ab = (a(ba)^{n-1}b)^{k}a = (ab)^{nk}a$ which is a reflection and not a translation, so it does not equal $ab$.\\
    \\
    Thus, we have that $ab \notin \im \alpha$, and so we know that if $\alpha$ takes $a$ and $b$ to be reflections over lines that are more that $1$ unit apart then $\alpha$ is not surjective, and so it is not an automorphism as desired.
\end{proof}
\noindent Part of the reason why these reflections have to be in this order is because no element of $D_\infty$ commutes with all elements other than the identity.

\begin{theorem}\label{conjugation}
    For all $g \in D_\infty$, $gx=xg$ if and only if $g = e$, or that $gxg^{-1} = x$ if and only if $g = e$.
\end{theorem}

\begin{proof}
    We want to show that the center of $D_\infty = \{e\}$. We know that $e$ is the reflection $x$, where every point reflects over itself. Let $x \in D_\infty$. Then $x$ is some combination of alternating $b$'s and $a$'s. Then we can see that $exe = x$ by definition. If $d \neq e$, then $d$ either ends in $a$ or $b$. Similarly, because $d^{-1}$ is the reverse of $d$, $d^{-1}$ starts with the same letter as $d$. So we can see that if $d$ ends with $a$, then $dbd^{-1}$ does not cancel at all, and we can see that $dbd^{-1} \neq b$. Similarly, if $d$ ends with $b$, then $dad^{-1}$ does not cancel, so we can see that $dad^{-1} \neq a$ as desired. Thus, we have that the center of $D_\infty$ is $\{e\}$ as desired.
\end{proof}

\noindent We can then see as a result of theorem $\ref{conjugation}$ that $D_\infty \cong Inn(D_\infty)$, where
\begin{center}
    $Inn(D_\infty) = \{\varphi_g: x \mapsto gxg^{-1} | g \in D_\infty\}$
\end{center}

\begin{center}
\fbox{\begin{minipage}{15cm}\textbf{Example:} Pick $ababa \in D_\infty$. We will then define the homomorphism $\phi: D_\infty \rightarrow Inn(D_\infty)$ to be such that $\phi: g \mapsto (\phi_g: x \mapsto gxg^{-1})$. Then we can see that $\phi(ababa) = \phi_{ababa}: x \mapsto ababaxababa$, so we can see that $\phi_{ababa}(a) = (ab)^4a$ and $\phi_{ababa}(b) = (ab)^5a$.\end{minipage}}
\end{center}

\begin{lemma}\label{conj}
    $D_\infty \cong Inn(D_\infty)$.
\end{lemma}

\begin{proof}
We want to construct an isomorphism $\varphi: D_\infty \rightarrow Inn(D_\infty)$, such that $a \mapsto (\varphi_a: x \mapsto axa^{-1})$, $b \mapsto (\varphi_b: x \mapsto bxb^{-1})$. First, we want to show that $\varphi$ is a homomorphism. Pick arbitrary $y,y' \in D_\infty$. Then we can see that 
\begin{center}
    $\varphi(yy') = (\varphi_{yy'}: x \mapsto yy'x(yy')^{-1}) = (\varphi_{yy'}: x \mapsto yy'xy'^{-1}y^{-1}) = \varphi_y \circ \varphi_{y'}(x) = \varphi(y)\varphi(y')$
\end{center}
 as desired. Next, we want to show that $\varphi$ is bijective. We can see that for all $g \in D_\infty$ that $\varphi(g) = \varphi_g: x \mapsto gxg^{-1}$, so we have that $\varphi$ is surjective because it hits every element in $Inn(D_\infty)$ as desired. Next, we want to show that $\varphi$ is injective. Assume for the sake of contradiction that for $g,g' \in D_\infty$ such that $g \neq g'$ that $\varphi(g) = \varphi(g')$. Then $gxg^{-1} = \varphi_g(x) = \varphi(g) = \varphi(g') = \varphi_{g'}(x) = g'xg'^{-1}$. Then we can see that $g'^{-1}gxg^{-1}g' = x$. But we know that the center of $x$ is $\{e\}$, so $g'^{-1}g = e$, and so $g = g'$. Thus, we have that $\varphi$ is injective, and so it is bijective as desired. Thus, we have an isomorphism  $\varphi: D_\infty \rightarrow Inn(D_\infty)$, and so we know that $D_\infty \cong Inn(D_\infty)$ as desired. Because we know that for all $g \in D_\infty$ that $\varphi_g$ is an automorphism, we can see that $Inn(D_\infty) \leq Aut(D_\infty)$, and so $D_\infty \cong Inn(D_\infty) < Aut(D_\infty)$.
\end{proof}

\noindent So we not only have that no element of $D_\infty$ commutes with all other elements of $D_\infty$ besides the identity, but also that $D_\infty \cong Inn(D_\infty)$ as a consequence. Further, we also showed that all $\alpha \in Inn(D_\infty)$ are automorphisms, so we can see that $Inn(D_\infty) \leq Aut(D_\infty)$. But exactly how many automorphisms are there that are not inner automorphisms? It turns out that that the answer is twice the number of inner automorphisms.

\begin{theorem}
    The index of $Inn(D_\infty)$ in $Aut(D_\infty)$ is $2$.
\end{theorem}

\begin{proof}
    We want to show that $|\{\varphi: G \rightarrow G | \varphi \text{isomorphism}\}: \{\varphi_g : G \rightarrow G | \varphi_g(x) = gxg^{-1}\}| = 2$. \\
\\
Define $\varphi: D_\infty \rightarrow D_\infty$ such that $\varphi(a) = b$ and $\varphi(b) = a$.\\
\\
We have proven that every conjugation map is an automorphism.
Then we have to show that $Aut(D_\infty) = Inn(D_\infty) \cup \varphi Inn(D_\infty)$ for some $\varphi$.
Pick $\psi \in Aut(D_\infty) - Inn(D_\infty)$. Then we can see that $\psi \circ \varphi(a) = \psi(b)$ and that $\psi \circ \varphi(b) = \psi(a)$. We want to show that this is a conjugation. Because $\psi \in Aut,$ we know by lemma $\ref{auto-fixation}$ that $\psi(a) = r_n$ and that $\psi(b) = r_{n \pm 1}$. Then we can see that $\psi(a) = (ba)^{n-1}b$ or $\psi(a) = (ab)^{n-1}a$ for some $n \in \mathbb{N}$. We can see that if $n-1$ is even, then $(ab)^{n-1}a = (ab)^{(n-1)/2}a(ba)^{(n-1)/2}$, so we know that $\psi(a) = (ba)^{n-1}b$ because $\psi(a)$ is not a conjugation, and so that $\psi(b) = (ab)^{n-1}a$. If $n-1$ is odd, then we can see that $\psi(a) = (ab)^{n-1}a$ and that $\psi(b) = (ab)^{n-1\pm1}a$, because the center letters are the opposite of the letter that is passed in. Then we can see that if $n-1$ is even, $\psi \o \varphi(a) = \psi(b) = (ab)^{n-1\pm1}a$. But since we know that $n-1\pm1$ is odd, we can see that $(ab)^{n-1\pm1}a = (ab)^{(n-1\pm1)/2}a(ba)^{(n-1\pm1)/2}$, which is a conjugation around $a$. We can see that this is similar for $\psi(b)$. We can also see that this holds if $n-1$ is odd. Thus, we have that $\psi \circ \varphi(x)$ is a conjugation of $x$, so $\psi \circ \varphi \in Inn(D_\infty)$.\\
\\
Then we can see that $\psi \circ \varphi \circ \varphi^{-1} \in Inn(D_\infty) \varphi^{-1} = Inn(D_\infty) \varphi$ because $\varphi = \varphi^{-1}$. Thus, we have that $Aut(D_\infty) = Inn(D_\infty) \cup Inn(D_\infty) \varphi$ as desired. So $|Aut : Inn| = 2$ as desired.
\end{proof}

\noindent Now we have that $D_\infty \cong Inn(D_\infty)$, and that $|Aut(D_\infty) : Inn(D_\infty)| = 2$. But strangely enough, it also turns out that $D_\infty \cong Aut(D_\infty)$. That is, for every reflection and translation, it turns out that there is an automorphism associated with it.
\begin{center}\fbox{\begin{minipage}{15cm}\textbf{Example:} Pick automorphism $\alpha$ such that $\alpha(x) = ba\psi(x)ab$, where $\psi: D_\infty \rightarrow D_\infty$ is such that $\psi: a \mapsto b$ and $b \mapsto a$. Then we can see that, using the $\phi$ defined in $\ref{conj}$, $\alpha(x) = \psi\circ\phi_a\circ\psi\circ\phi_a\circ\psi$, because $\psi\circ\phi_a\circ\psi\circ\phi_a\circ\psi = \psi\circ\phi_a\circ\psi\circ(\phi: x \mapsto a\psi(x)a) = \psi\circ\phi_a\circ(\phi: x \mapsto bxb) = \psi\circ(\phi: x \mapsto abxba) = \phi: x \mapsto ba\phi(x)ab$. So then, if we define $\varphi: Aut(D_\infty) \rightarrow D_\infty$ to be such that $\phi_a \mapsto a$ and $\psi \mapsto b$. Then we can see that $\varphi(\alpha) = \varphi(\psi)\varphi(\phi_a)\varphi(\psi)\varphi(\phi_a)\varphi(\psi) = babab$.\end{minipage}}\end{center}

\begin{center}

\end{center}

\begin{center}
    
\end{center}

\begin{theorem}
    $Aut(D_\infty) \cong D_\infty$.
\end{theorem}


\begin{proof} We want to construct an isomorphism $\varphi: Aut(D_\infty) \rightarrow D_\infty$. \\
    \\
Let $\psi: D_\infty \rightarrow D_\infty$ be such that $a \mapsto b$ and $b \mapsto a$, and $\phi: D_\infty \rightarrow D_\infty$ to be the isomorphism created in theorem \ref{conj}. We can see that every conjugation in $Inn(D_\infty)$ can be generated by $\phi_a$ and $\phi_b$. Then we can see that every $\alpha \in Aut(D_\infty)$ can be generated by $\phi_a,\phi_b$, and $\psi$. But then we can see that $\phi_a\psi = \psi\phi_b$, so we can see that we only need $\phi_a$ and $\psi$ in order to generate $Aut(D_\infty)$. So we have that $Aut(D_\infty) = \langle \phi_a, \psi \rangle$. Then we can construct a homomorphism $\varphi: Aut(D_\infty) \rightarrow D_\infty$ such that $\phi_a \mapsto a$ and $\psi \mapsto b$. \\
\\
We will proceed to show that $\varphi$ is well defined. Let $x_1x_2...x_n = x_1'x_2'...x_n'$. Then we can see that because the conjugations start with the same letter that $x_1 = x_1'$. We can continue this argument for each $x_i$ to see that $x_i = x_i'$ for all $i$. Then we can see that $\varphi(x_1x_2...x_n) = \varphi(x_1)\varphi(x_2)...\varphi(x_n) = \varphi(x_1')\varphi(x_2')...\varphi(x_n') = \varphi(x_1'x_2'...x_n')$ as desired. Thus, $\varphi$ is well defined, and so it is a homomorphism as desired.
Next, we want to show that $\varphi$ is surjective. Since $\varphi(\phi_a) = a$ and $\varphi(\psi) = b$, we can see that $a,b \in \im\varphi$, and so because $D_\infty$ is generated from $a$ and $b$ we can conclude that $D_\infty$ is generated from $\im\varphi$, so $\varphi$ is surjective.\\
\\
Lastly, we want to show that $\varphi$ is injective. Let $y = y_1y_2...y_n$ where $y_i \in \{a,b\}$, because $D_\infty$ is generated from $a$ and $b$. Then we can see that because only $\varphi(\phi_a)=a$ that if $y_i = a$ then $\varphi(x_i) = a$ and so $x_i = \phi_a$, and similarly that if $y_i = b$ then $\varphi(x_i) = b$ and so $x_i = \psi$. Thus, we have that there is only one pre-image for $y$, so $\varphi$ is injective as desired.
\end{proof}
\noindent One of the most interesting things about these reflections, however, is that they exist in not only one, but two dimensions. In fact, we can see that if we tiled the Euclidean plane with squares of side length 1 and let $W$ be the group of reflections between the squares, we can see that $W \cong D_\infty \oplus D_\infty$.\\
\\
For reflections over a square tiling of the plane, we let $a$ and $b$ represent reflections over the vertical lines $x=0$ and $x=1$ respectively, and $\alpha$ and $\beta$ be reflections over horizontal lines $y=0$ and $y=1$. $W$ is defined as the group generated by these elements, $W=\langle a,b\alpha ,\beta \rangle$.
\begin{theorem}
    Horizontal and vertical reflections commute with each other. 
\end{theorem}
\begin{proof}
Let $g$ be a horizontal reflection (a reflection over a vertical line), and let $\gamma$ be a vertical reflection (a reflection over a horizontal line). Then for some point $(x,y)$ in the plane, $g$ sends $(x,y)$ to $(x_1,y)$, and $\gamma$ sends $(x,y)$ to $(x,y_1)$. Thus $g\gamma$ first sends $(x,y)$ to $(x,y_1)$, then sends $(x,y_1)$ to $(x_1,y_1)$. $\gamma g$ first sends $(x,y)$ to $(x_1,y)$, then sends $(x_1,y)$ to $(x_1,y_1)$. Since $g\gamma$ and $\gamma g$ affect every point on the plane in the exact same way, they are the same element. So $g\gamma$ and $\gamma g$ were arbitrary horizontal and vertical reflections, horizontal and vertical reflections commute. Therefore $a\alpha =\alpha a$, $a\beta =\beta a$, $b\alpha =\alpha b$, and $b\beta = \beta b$.
\end{proof}
\begin{align*}
\begin{minipage}{7.5cm}
\begin{center}
    \includegraphics[height=1\textwidth]{Screen Shot 2023-05-26 at 12.52.48 PM.png}\\\caption{Figure 3: Caley Graph of $W$}
    \end{center}
\end{minipage} & \begin{minipage}{7.5cm}
\begin{center}
    \includegraphics[height=1\textwidth]{caley graph commutativity.pdf}\\
    \caption{Figure 4: Commutability of Reflections}
\end{center}
\end{minipage}
\end{align*}
\noindent For an example of the commutation of vertical and horizontal reflections, see figure 4. Note how regardless of path, when combining the reflections $\alpha$, $\beta$, and $b$, the end result is the same.
\\
\\
This commutation allows us to 'split up' the horizontal and vertical components of any element of $W$. Since the subgroups $\langle a, b\rangle $ and $\langle \alpha ,\beta \rangle$ are each a group of reflections over integers in one dimension, each should basically just be a copy of $D_\infty$. This can be seen directly on the Cayley Graph for $W$, shown in figure 3. The vertical and horizontal 'lines' which make up the Cayley Graph bear a striking resemblance to the Cayley Graph for $D_\infty$. This observation leads one to suspect that $W$ may be isomorphic to $D_\infty \times D_\infty$. 
\begin{theorem}
    $W\cong D_\infty \times D_\infty$
\end{theorem}
\begin{proof}
We can construct an isomorphism $\psi :W\rightarrow D_\infty \times D_\infty$. To start, we define two homomorphisms $\phi _1, \phi _2$ which both map $W\rightarrow D_\infty$. These are defined as follows:
\begin{align*}
    \phi _1(a)=a && \phi _1(b)=b && \phi _1(\alpha)=1 && \phi _1(\beta)=1
\end{align*}
and
\begin{align*}
    \phi_2(a)=1 && \phi _2(b)=1 && \phi_2(\alpha)=a && \phi_2(\beta)=b\\
\end{align*}
since we have defined how each function affects the generating elements of $W$, we can work out the effect on any other element using the fact that they are homomorphisms. Now we can define $\psi :W\rightarrow D_\infty \times D_\infty$ by 
\begin{equation*}
    \psi (x)=(\phi _1(x),\phi_2(x)).
\end{equation*}
First we will show that $\psi$ is a homomorphism. We have 
\begin{align*}
    \psi (x)\psi (y)&=(\phi _1(x),\phi _2(x))(\phi _1(y)\phi _2(y))\\
    &=(\phi _1(x)\phi _1(y),\phi_2(x)\phi_2(y))\\
    &=(\phi _1(xy),\phi _2(xy))\\
    &=\psi (xy)
\end{align*}
which just arises from the definition of the group operation in direct product groups and the fact that $\phi _1,\phi _2$ are both homomorphisms. \\
\\
Now we may show that $\psi$ is surjective. To do so, we can introduce two new homomorphisms $\varphi _1,\varphi_2:D_\infty \rightarrow W$. Their action on the generating elements of $D_\infty$ is 
\begin{align*}
    \varphi _1(a)=a && \varphi _1(b)=b && \varphi _2(a)=\alpha && \varphi _2(b)=\beta .
\end{align*}
Now let $(x,y)\in D_\infty \times D_\infty$ be some arbitrary element. Consider $\varphi_1(x)\varphi_2(y)$. $\varphi _1(x)$ contains only $a$'s and $b$'s, and $\varphi _2(y)$ contains only $\alpha$'s and $\beta$'s. Thus, $\phi _1(\varphi _1(x)\varphi _2(y))=x$ and $\phi _2(\varphi_1(x)\varphi _2(y))=y$. This means that $\psi (\varphi _1(x)\varphi _2(y))=(x,y)$, meaning $\varphi _1(x)\varphi _2(y)$ is a preimage for $(x,y)$ under $\psi$. Thus $\psi$ is surjective. \\
\\
To see that $\psi$ is injective, we can first note that since the vertical reflections commute with the horizontal ones, every element $x$ of $W$ has a unique representation $w\omega$, for which $w\in \langle a,b\rangle$ and $\omega \in \langle \alpha ,\beta \rangle$. Thus if for $x,y\in W$, $\psi (x)=\psi (y)$, we may write $x=w\omega$ and $y=v\nu$, where $w,v\in \langle a,b\rangle$ and $\omega \nu \in \langle \alpha, \beta \rangle$. So $\psi (w\omega )=\psi (v\nu)$. By the definition of $\psi$, $(\phi _1(w),\phi _2(\omega))=(\phi _1(v),\phi _2(\nu))$. From this we can read off that $\phi (w)=\phi (v)$. But both $w$ and $v$ are composed of $a$'s and $b$'s, and $\phi _1$ acts like the identity on these elements. So $w=v$. Also, we can tell that $\phi _2(\omega)=\phi _2(\nu)$, so $\varphi _2\circ \phi _2 (\omega)=\varphi _2\circ \phi _2 (\nu)$. But $\varphi _2\circ \phi _2$ acts like the identity on $\alpha$ and $\beta$, and $\omega$, $\nu$ are composed of only $\alpha$'s and $\beta$'s. So $\omega =\nu$, and so $w\omega =v\nu$, meaning $x=y$, and $\psi$ is an injective function. \\
\\
Now we have shown that $\phi :W\rightarrow D_\infty \times D_\infty$ is a bijective homomorphism, so it is an isomorphism. Thus $W\cong D_\infty \times D_\infty$.
\end{proof}
\section*{Works Cited}
Meier, John. “Groups Generated by Reflections.” \textit{Groups, Graphs and Trees an Introduction to the}

\textit{Geometry of Infinite Groups}, Cambridge University Press, Cambridge, 2008, pp. 44–53. 

\end{document}
